\documentclass[]{article}
\usepackage{lmodern}
\usepackage{amssymb,amsmath}
\usepackage{ifxetex,ifluatex}
\usepackage{fixltx2e} % provides \textsubscript
\ifnum 0\ifxetex 1\fi\ifluatex 1\fi=0 % if pdftex
  \usepackage[T1]{fontenc}
  \usepackage[utf8]{inputenc}
\else % if luatex or xelatex
  \ifxetex
    \usepackage{mathspec}
  \else
    \usepackage{fontspec}
  \fi
  \defaultfontfeatures{Ligatures=TeX,Scale=MatchLowercase}
\fi
% use upquote if available, for straight quotes in verbatim environments
\IfFileExists{upquote.sty}{\usepackage{upquote}}{}
% use microtype if available
\IfFileExists{microtype.sty}{%
\usepackage{microtype}
\UseMicrotypeSet[protrusion]{basicmath} % disable protrusion for tt fonts
}{}
\usepackage[margin=1in]{geometry}
\usepackage{hyperref}
\hypersetup{unicode=true,
            pdftitle={Assigment1},
            pdfauthor={Pavel Linder, Nikita Brancatisano},
            pdfborder={0 0 0},
            breaklinks=true}
\urlstyle{same}  % don't use monospace font for urls
\usepackage{graphicx,grffile}
\makeatletter
\def\maxwidth{\ifdim\Gin@nat@width>\linewidth\linewidth\else\Gin@nat@width\fi}
\def\maxheight{\ifdim\Gin@nat@height>\textheight\textheight\else\Gin@nat@height\fi}
\makeatother
% Scale images if necessary, so that they will not overflow the page
% margins by default, and it is still possible to overwrite the defaults
% using explicit options in \includegraphics[width, height, ...]{}
\setkeys{Gin}{width=\maxwidth,height=\maxheight,keepaspectratio}
\IfFileExists{parskip.sty}{%
\usepackage{parskip}
}{% else
\setlength{\parindent}{0pt}
\setlength{\parskip}{6pt plus 2pt minus 1pt}
}
\setlength{\emergencystretch}{3em}  % prevent overfull lines
\providecommand{\tightlist}{%
  \setlength{\itemsep}{0pt}\setlength{\parskip}{0pt}}
\setcounter{secnumdepth}{0}
% Redefines (sub)paragraphs to behave more like sections
\ifx\paragraph\undefined\else
\let\oldparagraph\paragraph
\renewcommand{\paragraph}[1]{\oldparagraph{#1}\mbox{}}
\fi
\ifx\subparagraph\undefined\else
\let\oldsubparagraph\subparagraph
\renewcommand{\subparagraph}[1]{\oldsubparagraph{#1}\mbox{}}
\fi

%%% Use protect on footnotes to avoid problems with footnotes in titles
\let\rmarkdownfootnote\footnote%
\def\footnote{\protect\rmarkdownfootnote}

%%% Change title format to be more compact
\usepackage{titling}

% Create subtitle command for use in maketitle
\providecommand{\subtitle}[1]{
  \posttitle{
    \begin{center}\large#1\end{center}
    }
}

\setlength{\droptitle}{-2em}

  \title{Assigment1}
    \pretitle{\vspace{\droptitle}\centering\huge}
  \posttitle{\par}
    \author{Pavel Linder, Nikita Brancatisano}
    \preauthor{\centering\large\emph}
  \postauthor{\par}
      \predate{\centering\large\emph}
  \postdate{\par}
    \date{11/6/2019}


\begin{document}
\maketitle

\hypertarget{installing-packages}{%
\subsection{Installing packages}\label{installing-packages}}

install.packages(``rgl'') install.packages(``caret'')
install.packages(``generics'') install.packages(``plot3D'')
install.packages(``gower'')

library(caret) library(``rgl'') library(``plot3D'') library(``GA'')

library(``dplyr'')

\hypertarget{what-does-it-mean-for-a-function-to-be-non-convex}{%
\subsection{What does it mean for a function to be
non-convex?}\label{what-does-it-mean-for-a-function-to-be-non-convex}}

\hypertarget{a-function-is-non-convex-or-concave-if-for-any-xx-and-yy-in-the-interval-and-for-any}{%
\section{\texorpdfstring{A function is non-convex (or concave) if for
any \{\displaystyle x\}x and \{\displaystyle y\}y in the interval and
for
any}{A function is non-convex (or concave) if for any \{x\}x and \{y\}y in the interval and for any}}\label{a-function-is-non-convex-or-concave-if-for-any-xx-and-yy-in-the-interval-and-for-any}}

\hypertarget{the-following-equation-is-valid}{%
\section{\texorpdfstring{\{\displaystyle \alpha \in [0,1]\}\alpha \in [0,1],
the following equation is
valid:}{\{\}, the following equation is valid:}}\label{the-following-equation-is-valid}}

\hypertarget{f1-xy1-fxfy}{%
\section{\texorpdfstring{\{\displaystyle f((1-\alpha )x+\alpha y)\geq (1-\alpha )f(x)+\alpha f(y)\}}{\{f((1-)x+y)(1-)f(x)+f(y)\}}}\label{f1-xy1-fxfy}}

\hypertarget{task-1-maximization-of-a-non-convex-function.}{%
\subsection{Task 1: Maximization of a non-convex
function.}\label{task-1-maximization-of-a-non-convex-function.}}

\hypertarget{code-a-method-fx-y-that-computes-a-value-z-given-an-input-tuple-x-y.}{%
\section{1. Code a method f(x, y) that computes a value z, given an
input tuple (x,
y).}\label{code-a-method-fx-y-that-computes-a-value-z-given-an-input-tuple-x-y.}}

f \textless- function(x, y) \{ z \textless- ((1 - x)\^{}2) + (exp(1) *
(y - (x\textsuperscript{2))}2) \}

\hypertarget{code-a-method-that-visualizes-the-rosenbrock-function-in-3d}{%
\section{2. Code a method that visualizes the Rosenbrock function in
3D}\label{code-a-method-that-visualizes-the-rosenbrock-function-in-3d}}

x \textless- y \textless- seq(-1, 1, length= 20) z \textless- outer(x,
y, f) z{[}is.na(z){]} \textless- 1 \# change non-defined elements to 1
persp(x, y, z, theta = 30, phi = 20, expand = 1, col = ``lightblue'',
ticktype = ``detailed'')

\hypertarget{code-a-genetic-algorithm-that-attempts-to-find-the-global-maximum-of-this-function.}{%
\section{3. Code a genetic algorithm, that attempts to find the global
maximum of this
function.}\label{code-a-genetic-algorithm-that-attempts-to-find-the-global-maximum-of-this-function.}}

crossovers = c(``ga\_spCrossover'', ``gabin\_spCrossover'',
``gabin\_uCrossover'', ``gareal\_spCrossover'', ``gareal\_waCrossover'',
``gareal\_laCrossover'', ``gareal\_blxCrossover'',
``gareal\_laplaceCrossover'', ``gaperm\_cxCrossover'',
``gaperm\_pmxCrossover'', ``gaperm\_pmxCrossover'' ) results = c() best
\textless- 0 name \textless- 0 for (i in 1:length(crossovers)) \{ GA
\textless- ga(type = ``real-valued'', fitness = function(x) f(x{[}1{]},
x{[}2{]}), lower = c(-1, -1), upper = c(1, 1), popSize = 50, maxiter =
1000, run = 100, crossover = crossovers{[}i{]} ) if (best \textless{}
\href{mailto:GA@fitness}{\nolinkurl{GA@fitness}}) \{ name \textless-
crossovers{[}i{]} best{[}1{]} \textless-
\href{mailto:GA@fitness}{\nolinkurl{GA@fitness}} \} results{[}i{]}
\textless- \href{mailto:GA@fitness}{\nolinkurl{GA@fitness}} \}

results best name

plot(GA) summary(GA) res \textless- persp3D(x = x, y = x, z = z,theta =
30, phi = 25, col.palette = bl2gr.colors)

max\_x = \href{mailto:GA@solution}{\nolinkurl{GA@solution}}{[}1{]}
max\_y = \href{mailto:GA@solution}{\nolinkurl{GA@solution}}{[}2{]}
max\_z = f(\href{mailto:GA@solution}{\nolinkurl{GA@solution}}{[}1{]},
\href{mailto:GA@solution}{\nolinkurl{GA@solution}}{[}2{]})

points3D(max\_z, max\_y, max\_z, col = ``black'', size = 30, add=T)

\hypertarget{bonus-plot-the-trace-of-evolution}{%
\section{BONUS: Plot the trace of
evolution}\label{bonus-plot-the-trace-of-evolution}}

history\_x = c() history\_y = c()

for (i in 1:12) \{ GA2 \textless- ga(type = ``real-valued'', fitness =
function(x) f(x{[}1{]}, x{[}2{]}), lower = c(-1, -1), upper = c(1, 1),
popSize = 50, maxiter = 25*i, run = 100) history\_x{[}i{]} \textless-
\href{mailto:GA2@solution}{\nolinkurl{GA2@solution}}{[},1{]}
history\_y{[}i{]} \textless-
\href{mailto:GA2@solution}{\nolinkurl{GA2@solution}}{[},2{]} \}

res2 \textless- persp3D(x = x, y = y, z = z,theta = 30, phi = 25,
col.palette = bl2gr.colors, ticktype = ``detailed'') points3D(x =
history\_x, y = history\_y, z = f(history\_x, history\_y), col =
``black'', size = 2, transparency = 0, add=T) text3D(x =
history\_x{[}1:3{]}, y = history\_y{[}1:3{]}, z = f(history\_x{[}1:3{]},
history\_y{[}1:3{]}), labels = 25\emph{c(1:3), col = ``black'', pos=4,
size = 0.1, add=T) text3D(x = history\_x{[}10:12{]}, y =
history\_y{[}10:12{]}, z = f(history\_x{[}10:12{]},
history\_y{[}10:12{]}), labels = 25}c(10:12), col = ``black'', size =
0.1, pos=4, add=T)

\hypertarget{discuss-the-results.}{%
\subsection{4. Discuss the results.}\label{discuss-the-results.}}

\hypertarget{how-does-performance-vary-when-you-are-increasing-the-number-of-iterations}{%
\subsection{• How does performance vary when you are increasing the
number of
iterations?}\label{how-does-performance-vary-when-you-are-increasing-the-number-of-iterations}}

\hypertarget{the-performance-is-slowed-down-because-the-ga-reaches-the-maximum-solution-much-before-we-run-all-of-the-iterations-thus}{%
\section{The performance is slowed down because the GA reaches the
maximum solution much before we run all of the iterations,
thus}\label{the-performance-is-slowed-down-because-the-ga-reaches-the-maximum-solution-much-before-we-run-all-of-the-iterations-thus}}

\hypertarget{increasing-the-number-would-only-slow-down-the-search-without-providing-any-significant-benefit.}{%
\section{increasing the number would only slow down the search without
providing any significant
benefit.}\label{increasing-the-number-would-only-slow-down-the-search-without-providing-any-significant-benefit.}}

\#\#• What about population size? \# Increasing the population size
actually speeds up the process but there's a diminishing return on how
much we can increase the \# population. After a certain amount
(depending on the problem) increasing it just slows down the GA.

\#\#• Explain the difference between local and global maxima. \# A local
maxima is the biggest element for one given subset of the whole
function. There can still be other values greater than \# this one,
outside ofthe range of the subset. The global maximum is the largest
overall value of the funtion.

\#\#Task 2: Genetic feature selection \# Work plan: \# 1) Read in the
data \# 2) Split the data into training and test sets \# 3) Build a
model using the training set \# 4) Evaluate the model using the test set

\hypertarget{read-in-the-dataset}{%
\section{1) Read in the dataset}\label{read-in-the-dataset}}

genes \textless- DLBCL summary(genes) target \textless- genes\$class
\#genes \textless- genes{[},-which(names(genes) == ``class''){]}

\hypertarget{split-the-data-into-training-and-test-sets}{%
\section{2) Split the data into training and test
sets}\label{split-the-data-into-training-and-test-sets}}

splited \textless- createDataPartition( y = target, p = .8, list = FALSE
)

str(splited) \# The output is a set of integers for the rows of genes
that belong in the training set. learn = genes{[} splited, {]} test =
genes{[}-splited, {]}

nrow(learn) nrow(test) table(learn\(class) table(test\)class)
learn\(X <- NULL test\)X \textless- NULL

\hypertarget{build-a-model-using-the-training-set}{%
\section{3) Build a model using the training
set}\label{build-a-model-using-the-training-set}}

\hypertarget{todo-fitness-must-take-into-account-both-the-number-of-features-as-well-as-the-classifiers-performance}{%
\section{(TODO: • Fitness must take into account both the number of
features, as well as the classifier's
performance)}\label{todo-fitness-must-take-into-account-both-the-number-of-features-as-well-as-the-classifiers-performance}}

ctrl \textless- trainControl(number = 3, method = ``repeatedcv'')

PLSModel \textless- train( class \textasciitilde{} ., data = learn,
method = ``pls'', preProc = c(``center'', ``scale''), trControl = ctrl,
tuneLenght = 1000 )

RFModel \textless- train( class \textasciitilde{} .,\\
data = learn, method = ``rf'', preProc = c(``center'', ``scale''),
trControl = ctrl, tuneLenght = 1000 )

xgbModel \textless- train( class \textasciitilde{} .,\\
data = learn, method = ``xgbDART'', preProc = c(``center'', ``scale''),
trControl = ctrl, tuneLenght = 100 )

SVMModel \textless- train( class \textasciitilde{} .,\\
data = learn, method = ``svmRadial'', preProc = c(``center'',
``scale''), trControl = ctrl, tuneLenght = 1000 )

models\_compare \textless- resamples(list(RF=RFModel, XGBDART=xgbModel,
PLS=PLSModel, SVM=SVMModel))

summary(models\_compare)

varimp \textless- varImp(RFModel) imp \textless- varimp\$importance

plot(varimp)

\hypertarget{evaluate-the-model-using-the-test-set-todo-try-different-models}{%
\section{4) Evaluate the model using the test set (TODO: try different
models)}\label{evaluate-the-model-using-the-test-set-todo-try-different-models}}

plsClasses \textless- predict(xgbModel, newdata = test) str(plsClasses)

plsProbs \textless- predict(xgbModel, newdata = test, type = ``prob'')
head(plsProbs)

confusionMatrix(data = plsClasses, test\$class)


\end{document}
